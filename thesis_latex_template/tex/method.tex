\chapter{METHODOLOGY}
\label{ch:methodology}

\section{System Requirements}
\label{sec:system-requirements}

\subsection{Functional Requirements}

\begin{table}[H]
\centering
\caption{Functional Requirements}
\label{tab:functional-requirements}
\begin{tabular}{|p{1cm}|p{8cm}|p{2.5cm}|}
\hline
\textbf{ID} & \textbf{Requirement} & \textbf{Priority} \\
\hline
FR1 & Requirement 1 & Must \\
\hline
FR2 & Requirement 2 & Should \\
\hline
\end{tabular}
\end{table}

\subsection{Non-Functional Requirements}

\begin{table}[H]
\centering
\caption{Non-Functional Requirements}
\label{tab:non-functional-requirements}
\begin{tabular}{|p{1.5cm}|p{5cm}|p{5.5cm}|}
\hline
\textbf{ID} & \textbf{Requirement} & \textbf{Target} \\
\hline
NFR1 & Performance & Less than X seconds \\
\hline
NFR2 & Cost & Less than \$X \\
\hline
\end{tabular}
\end{table}

\section{System Design}
\label{sec:system-design}

% Add figures to illustrate your design
% \begin{figure}[H]
% \centering
% \caption{System Architecture}
% \label{fig:system-architecture}
% \includegraphics[width=0.95\textwidth]{figures/system_overview.png}
% \end{figure}

\section{Complexity Analysis}

\begin{table}[H]
\centering
\caption{Time Complexity Analysis}
\label{tab:time-complexity}
\begin{tabular}{|p{4.5cm}|p{3cm}|p{5.5cm}|}
\hline
\textbf{Operation} & \textbf{Complexity} & \textbf{Notes} \\
\hline
Operation 1 & O(n) & Description \\
\hline
Operation 2 & O(log n) & Description \\
\hline
\end{tabular}
\end{table}
